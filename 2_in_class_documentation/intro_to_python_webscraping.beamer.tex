% Created 2014-05-22 Thu 13:25
\documentclass[presentation,smaller]{beamer}
\usepackage[utf8]{inputenc}
\usepackage[T1]{fontenc}
\usepackage{fixltx2e}
\usepackage{graphicx}
\usepackage{longtable}
\usepackage{float}
\usepackage{wrapfig}
\usepackage{rotating}
\usepackage[normalem]{ulem}
\usepackage{amsmath}
\usepackage{textcomp}
\usepackage{marvosym}
\usepackage{wasysym}
\usepackage{amssymb}
\usepackage{hyperref}
\tolerance=1000
\usepackage{color}
\usepackage{listings}
\usepackage{listings}
\usepackage{color}

\definecolor{mygreen}{rgb}{0,0.6,0}
\definecolor{mygray}{rgb}{0.5,0.5,0.5}
\definecolor{mymauve}{rgb}{0.58,0,0.82}

\lstset{ %
  numbers=left,                    % where to put the line-numbers; possible values are (none, left, right)
  numbersep=5pt,                   % how far the line-numbers are from the code
  numberstyle=\color{black}, % the style that is used for the line-numbers
  backgroundcolor=\color{white},   % choose the background color; you must add \usepackage{color} or \usepackage{xcolor}
  %basicstyle=\footnotesize,        % the size of the fonts that are used for the code
  basicstyle=\scriptsize,  
breakatwhitespace=true,         % sets if automatic breaks should only happen at whitespace
  breaklines=true,                 % sets automatic line breaking
  captionpos=b,                    % sets the caption-position to bottom
  commentstyle=\color{mygreen},    % comment style
  deletekeywords={...},            % if you want to delete keywords from the given language
  escapeinside={\%*}{*)},          % if you want to add LaTeX within your code
  extendedchars=true,              % lets you use non-ASCII characters; for 8-bits encodings only, does not work with UTF-8
  frame=single,                    % adds a frame around the code
  keepspaces=true,                 % keeps spaces in text, useful for keeping indentation of code (possibly needs columns=flexible)
  keywordstyle=\color{blue},       % keyword style
  %language=Octave,                 % the language of the code
  morekeywords={*,...},            % if you want to add more keywords to the set
  rulecolor=\color{black},         % if not set, the frame-color may be changed on line-breaks within not-black text (e.g. comments (green here))
  showspaces=false,                % show spaces everywhere adding particular underscores; it overrides 'showstringspaces'
  showstringspaces=false,          % underline spaces within strings only
  showtabs=false,                  % show tabs within strings adding particular underscores
  stepnumber=1,                    % the step between two line-numbers. If it's 1, each line will be numbered
  stringstyle=\color{mymauve},     % string literal style
  tabsize=2,                       % sets default tabsize to 2 spaces
  title=\lstname                   % show the filename of files included with \lstinputlisting; also try caption instead of title
}


\usepackage{attachfile2}
\usetheme{default}
\author{Robert Vesco}
\date{\today}
\title{Pre-class Installation Requirements for \\ Intro to Python and Webscraping}
\hypersetup{
  pdfkeywords={},
  pdfsubject={},
  pdfcreator={Emacs 24.3.1 (Org mode 8.2.5h)}}
\begin{document}

\maketitle
\begin{frame}{Outline}
\tableofcontents
\end{frame}


\begin{frame}[label=sec-1]{Introduction}
\end{frame}


\begin{frame}[label=sec-2]{How to Share Ipython Notebooks}
\end{frame}


\begin{frame}[label=sec-3]{How to share your vagrant box}
\end{frame}


\begin{frame}[fragile,label=sec-4]{How to use virtualenv \& pip}

 Virtualenv allow you to create environments. 

\lstset{numbers=left,language=sh}
\begin{lstlisting}
## run this on the command line

## assuming you are in your projects folder, create a new folder

mkdir projects1 

cd projects1

## now create your virtualenv environment
## this will create a folder called "env". 
## this will house a local version of python. 

virtualenv env 

## IMPORTANT. 
## Now you need to activate your environment. 

source env/bin/activate

## now you will be using a local version of python instead of your
## system's python

## to deactivate, simply type

deactivate
\end{lstlisting}
\end{frame}

\begin{frame}[fragile,label=sec-5]{How to use virtualenv * pip}
 Now once virtualenv is installed, you can start installing modules
locally. 

The program to do this is called pip. New versions of python may come
with it already installed, but older version may require manual 
installation. 

\lstset{numbers=left,language=sh}
\begin{lstlisting}
## again, these should be run on the command line. 

## first, let's activate your virtual environment, if you haven't 
## already

source env/bin/activate

## first, let's inspect what command are available in pip

pip help

## from this, we see that there are a number of commands we will 
## find useful

pip list # this shows what programs are already installed

pip search numpy # this searches for packages named "numpy"

pip install numpy # this installs the numpy package. 

## if you have many packages you want to install, you can 
## create a requirements list

## this will create a file with a list of modules to install
## you can use your editor of choice to install this. 

echo "numpy\nbeautifulsoup" > requirements.txt

## this will install all the packages in the text file. 
## NOTE: you can specify the versions of module too. Sometimes
## this is important. 

pip install -r requirements.txt

## now let's confirm that they installed correctly

pip list 

## now if you are done with virtualenv remember to deactivate it

deactivate
\end{lstlisting}
\end{frame}
% Emacs 24.3.1 (Org mode 8.2.5h)
\end{document}