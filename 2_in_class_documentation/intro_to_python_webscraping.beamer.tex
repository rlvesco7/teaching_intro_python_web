% Created 2014-05-26 Mon 06:35
\documentclass[presentation]{beamer}
\usepackage[utf8]{inputenc}
\usepackage[T1]{fontenc}
\usepackage{fixltx2e}
\usepackage{graphicx}
\usepackage{longtable}
\usepackage{float}
\usepackage{wrapfig}
\usepackage{rotating}
\usepackage[normalem]{ulem}
\usepackage{amsmath}
\usepackage{textcomp}
\usepackage{marvosym}
\usepackage{wasysym}
\usepackage{amssymb}
\usepackage{hyperref}
\tolerance=1000
\usepackage{color}
\usepackage{listings}
\input{preamble.tex}
\usepackage{attachfile2}
\setbeamertemplate{itemize/enumerate subbody begin}{\vspace{0.1cm}}
\setbeamertemplate{itemize/enumerate subbody end}{\vspace{0.1cm}}
\usetheme{Frankfurt}
\usecolortheme{beaver}
\usefonttheme{professionalfonts}
\author{Robert Vesco}
\date{\today}
\title{Intro to Python and Webscraping}
\hypersetup{
  pdfkeywords={},
  pdfsubject={},
  pdfcreator={Emacs 24.3.1 (Org mode 8.2.5h)}}
\begin{document}

\maketitle

\section{Intro}
\label{sec-1}

\begin{frame}[label=sec-1-1]{Class Objectives}
\begin{itemize}
\item Introduce basic python and webscraping
\item Provide skills \& knowledge not in online tutorials
\item Tools that can be used with any programming language
\end{itemize}
\end{frame}
\begin{frame}[label=sec-1-2]{Plan}
\begin{itemize}
\item 9 - 9:15: Setup issues
\item 9:15 - 9:30 Python in Scientific Computing
\item 9:30 - 9:45 Anaconda \& Spyder
\item 9:45 - 10:30 Command line basics
\item 10:30 - 12:00 Python Basics
\item 12:00 - 12:30 Lunch
\item 12:30 - 3:00 Python Webscraping
\item 3:00 - 4:30 Practice with your own site
\item 4:30 - 5:00 Other Tools, Development Environment
\end{itemize}
\end{frame}
\section{Python Context}
\label{sec-2}

\begin{frame}[label=sec-2-1]{Abbreviated/Opinionated History of Programming Languages}
\begin{itemize}
\item C, C++
\item Awk, Sed \& shell scripts
\item Practical Extraction and Reporting (perl)
\item S (R precursor)
\item Java
\item Ruby (perl 2.0)
\item R
\item Python
\item Julia (R 2.0)
\end{itemize}
\end{frame}
\begin{frame}[label=sec-2-2]{Python and Stats}
\end{frame}

\begin{frame}[label=sec-2-3]{Python and Jobs}
\end{frame}

\begin{frame}[label=sec-2-4]{Python Considerations}
\begin{columns}
\begin{column}{0.5\textwidth}
\begin{block}{Support For}
\begin{itemize}
\item Readability \& Consistency (pythonic)
\item Fairly fast
\item Not Java
\item Used in biz ops \& domains
\end{itemize}
\end{block}
\end{column}
\begin{column}{0.5\textwidth}
\begin{block}{Support Against}
\begin{itemize}
\item Backward compatibility
\item Fragile package dependencies
\item Fragmentation
\item Complementary Assets for Science
\end{itemize}
\end{block}
\end{column}
\end{columns}
\end{frame}
\begin{frame}[label=sec-2-5]{The many faces and versions of Python}
\begin{itemize}
\item Cython (main)
\item IronPython (.net)
\item PyPy (JIT)
\item Jython (compiles to java)
\item Ipython (scientific and interactive)
\end{itemize}
\end{frame}
\section{Terminals}
\label{sec-3}

\begin{frame}[fragile,label=sec-3-1]{Anaconda and Spydyer}
 \begin{itemize}
\item Anaconda is a pre-packaged python distribution for scientists
\item Spyder is an IDE (Integrated Development Environment)
\item Open a terminal or click spyder
\end{itemize}

\lstset{numbers=left,language=sh}
\begin{lstlisting}
anaconda/bin/spyder
\end{lstlisting}

\begin{itemize}
\item Open terminal within spyder
\end{itemize}
\end{frame}

\begin{frame}[label=sec-3-2]{Why Terminals and Command Line Programs?}
\begin{itemize}
\item Troubleshooting python programs
\item Managing programs and files
\item Right tool for some jobs
\end{itemize}
\end{frame}
\begin{frame}[label=sec-3-3]{Shells vs Terminals}
\begin{itemize}
\item Shells are programs (like python) that help you interact computer.
\begin{itemize}
\item csh (c shell, mostly seen on older servers)
\item bash (most common)
\item zsh (most convenient)
\end{itemize}
\item Terminals are wrappers around shells (iterm2 for macs)
\item .bashrc, .cshrc, .zshrc are configuration files for shells
\end{itemize}
\end{frame}
\begin{frame}[label=sec-3-4]{}
\end{frame}
\section{Python Basics}
\label{sec-4}
\section{Webscraping}
\label{sec-5}

\begin{frame}[label=sec-6-1]{Top Aligned Blocks}
\begin{columns}
\begin{column}{0.5\textwidth}
\begin{block}{Code}
Cool
Lots
of Stuf

To talk

about
\end{block}
\end{column}
\begin{column}{0.5\textwidth}
\begin{block}{Result}
pretty nice!
\end{block}
\end{column}
\end{columns}
\end{frame}

\begin{frame}[label=sec-6-2]{Inline math}
\end{frame}


\begin{frame}[label=sec-6-3]{Beamer: Animated Bullets}
\begin{itemize}[<+->]
\item Trouble Shooting
\item A framework for thinking about programming
\end{itemize}
\end{frame}

\begin{frame}[label=sec-6-4]{Beamer Columns}
\begin{columns}
\begin{column}{0.5\textwidth}
\begin{block}{Stuff}
\begin{itemize}
\item Truth is ephemeral
\end{itemize}
\end{block}
\end{column}

\begin{column}{0.5\textwidth}
\begin{itemize}
\item What is right?
\item What is Wrong?
\end{itemize}
\end{column}
\end{columns}
\end{frame}



\begin{frame}[fragile,label=sec-7-1]{How to use virtualenv \& pip}
 \lstset{numbers=left,language=sh}
\begin{lstlisting}
## run this on the command line
## assuming you are in your projects folder, create a new folder
mkdir projects1 

cd projects1

## now create your virtualenv environment
## this will create a folder called "env". 
## this will house a local version of python. 
virtualenv env 

## IMPORTANT. 
## Now you need to activate your environment. 
source env/bin/activate

## now you will be using a local version of python instead of your
## system's python

## to deactivate, simply type
deactivate
\end{lstlisting}
\end{frame}
\begin{frame}[label=sec-7-2]{How to Share Ipython Notebooks}
\end{frame}

\begin{frame}[label=sec-7-3]{How to share your vagrant box}
\end{frame}
\begin{frame}[fragile,label=sec-7-4]{Testing Python Output}
 \lstset{numbers=left,language=Python}
\begin{lstlisting}
a = ('b', 200)
b = ('x', 10)
c = ('q', -42)
return (a, b, c)
\end{lstlisting}

\begin{center}
\begin{tabular}{lr}
b & 200\\
x & 10\\
q & -42\\
\end{tabular}
\end{center}
\end{frame}

\begin{frame}[fragile,label=sec-7-5]{Python Output}
 \lstset{numbers=left,language=Python}
\begin{lstlisting}
a = ('b', 200)
b = ('x', 10)
c = ('q', -42)
return (a, b, c)
\end{lstlisting}

By removing the :exports both, you can export just the code and not the output. By replaceing it with :exports results, you can export the output without the source. 
\end{frame}

\begin{frame}[fragile,label=sec-7-6]{Using pip once virtualenv is activated}

 \lstset{numbers=left,language=sh}
\begin{lstlisting}
## again, these should be run on the command line. 
## first, let's activate your virtual environment, if you haven't 
## already
source env/bin/activate

## first, let's inspect what command are available in pip
pip help

## from this, we see that there are a number of commands we will 
## find useful
pip list # this shows what programs are already installed
pip search numpy # this searches for packages named "numpy"
pip install numpy # this installs the numpy package. 

## if you have many packages you want to install, you can 
## create a requirements list
## this will create a file with a list of modules to install
## you can use your editor of choice to install this. 
echo "numpy\nbeautifulsoup" > requirements.txt

## this will install all the packages in the text file. 
## NOTE: you can specify the versions of module too. Sometimes
## this is important. 
pip install -r requirements.txt

## now let's confirm that they installed correctly
pip list 

## now if you are done with virtualenv remember to deactivate it
deactivate
\end{lstlisting}
\end{frame}
% Emacs 24.3.1 (Org mode 8.2.5h)
\end{document}